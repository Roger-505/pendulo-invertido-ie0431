\section{Aclaración con respecto a la entrega de esta evaluación}
Hola profesor.
Previamente le había enviado un correo solicitando tiempo extra para entregar
esta evaluación, debido a que, a pesar de que ya había logrado simulado exitosamente
el sistema de control que me había planteado diseñar, ocupaba más tiempo para realizar
el informe escrito.
Usted me había concedido el tiempo extra de acuerdo al correo sin una fecha límite definida, pero debido a todas las evaluaciones
de final de semestre, estuve escazo de tiempo para terminar este informe (en particular, un informe de 80 páginas de un proyecto de otro curso
que estoy llevando que me tomó todo un fin de semana en completar).
Debido a eso, no entregué el informe de este trabajo para el momento en que se revisó el examen \#2.
Con suerte, logré despejarme de todas las evaluaciones de los otros cursos, a costo de haber puesto el informe de este examen
hasta el final de la cola de cosas que tenía que hacer a final de semestre.
Espero que pueda entender esta situación, y estoy atento ante las medidas que usted considere pertinentes 
tomar para solventarla.
A continuación, se muestra una versión muy corta, muy al grano, y (hice mi mejor esfuerzo para que lo fuera) muy buena del reporte para esta evaluación.

\section{Modelado de un sistema dinámico entre presa y depredador}
El esquema del sistema a modelar se muestra en la Figura 1.
Él consta de un sistema de péndulo invertido de brazo rotatorio, incluyendo actuador y sensores
para medir el cambio de los ángulos de rotación del brazo rotatorio y del péndulo.
Las señales del sistema son descritas en la \hyperref[t1]{Tabla 1}.

\begin{table}
\centering
\begin{tabular}{ccccp{3cm}}
\toprule
Señal &  Unidades & Rango de valores & Valor nominal & Descripción\\
\midrule
$\Delta \hat{\alpha}(t)$ & mA & $[4,\, 20]$ & 12& Valor sensado de $\alpha(t)$\\
$\Delta \hat{\theta}(t)$ & mA & $[ 4,\, 20]$ & 12& Valor sensado de $\theta(t)$\\
$\widetilde{V_m}(t)$ & V & $[0,\, 5]$ & 2.5       & Variable manipulada\\
$\alpha(t)$ & rad & $[ 5\pi/6,\, 7\pi/6]$ & $\pi$                           & Ángulo del péndulo con respecto al eje $-z$\\
$\theta(t)$ & rad & $[-\pi/2,\,\pi/2]$ & 0& Ángulo del brazo rotatorio medido con respecto al eje $x$\\
$V_m(t)$ & V & $[-10,\, 10]$ & 0& Tensión recibida por el motor DC del sistema\\
$\tau(t)$ & $\text{N}\cdot\text{m}$ & $[0,\,0.037]$& 0 & Torque aplicado al péndulo rotatorio. Perturbación\\
\bottomrule
\end{tabular}
\caption{Señales del sistema por controlar}
\label{t1}
\end{table}

El modelado se realizó utilizando el demo de Simulink \href{https://la.mathworks.com/matlabcentral/fileexchange/84175-inverted-pendulum-simscape-model-and-controller-design?requestedDomain=}{Inverted Pendulum Simscape Model and Controller design}.
El valor nominal de $\alpha(t)$ y $\theta(t)$ mostrados en la \hyperref[t1]{Tabla 1} son los que ya venían configurados en el demo.
$t$ es el tiempo, cuyas unidades son segundos.
Por simplicidad, se supondrá que ambos sensores son equivalentes.
Se plantearon los siguietnes objetivos para el sistema de control:
\begin{itemize}
    \item Cero error en estado estacionario ante cambios en la perturbación y en el valor deseado.
    \item Menor tiempo de asentamiento a lazo cerrado en comparación a lazo abierto.
\end{itemize}

Debido a la alto grado de no linealidad del sistema en puntos de operación con $\alpha(t) \neq \pi \wedge \alpha(t) \neq 0$, no se consideraron cambios en el valor deseado, de tal forma que se consideraron valores deseados constantes. 

\subsection{Diseño del actuador para el péndulo invertido rotatorio}
Se realizó una identificación preeliminar del sistema sin actuador ni sensores.
Para $P_{\alpha}(s)$, la constante de tiempo asociada a esta función de transferencia se muestra
en \eqref{cteTiempo}.
\begin{equation}
    T_{\alpha} = 0.12454 \label{cteTiempo}
\end{equation}
Esta es la constante de tiempo dominante del sistema, ya que $P_{\theta}(s)$ es integrante.
En base a esto, se modeló la dinámica de los sensores y actuador como funciones de transferencia
de primer orden con constantes de tiempo \eqref{ctesTiempo}.
\begin{equation}
    \begin{aligned}
        T_{sensor} = \frac{T_{\alpha}}{50}\label{ctesTiempo}\\
        T_{actuador} = \frac{T_{\alpha}}{50}
    \end{aligned}
\end{equation}
Además, se escojieron los puntos de operación mostrados en el \hyperref[t1]{Cuadro 1},
con lo cual se normalizó cada una de las señales por medio de las relaciones \eqref{relaciones}.
\begin{equation}
    \begin{aligned}
        V_m(t) &= 4\widetilde{V_m}(t) - 10\label{relaciones}\\
        \Delta\widehat{\alpha}(t) &= \frac{ \SI{48e-3}{}}{\pi}\alpha(t) + \SI{12e-3}{} -\pi\\
        \Delta\widehat{\theta}(t) &= \frac{ \SI{16e-3}{}}{\pi}\theta(t) + \SI{12e-3}{}
    \end{aligned}
\end{equation}
Por tanto, las ganancias de los sensores tendrán unidades A/rad, mientras que el actuador tendrá unidades p.u, ya 
que recibe y entrega señales de tensión.

\subsection{Identificación del sistema}
Al ser el sistema por controlar tipo SIMO, se identificaron dos plantas, $P_{\alpha}(s)$ y
$P_{\theta}(s)$ que satisfacen la relación \eqref{ident}.
\begin{equation}
    \begin{aligned}
        P_{\alpha}(s) = \frac{\Delta\hat{\alpha}(s)}{\widetilde{V_m}(s)}\Bigg|_{\theta(t)=0}\label{ident}\\
        P_{\theta}(s) = -\frac{\Delta\hat{\theta}(s)}{\widetilde{V_m}(s)}\Bigg|_{\alpha(t)=\pi}
    \end{aligned}
\end{equation}
    $\forall t > \SI{0}{s}$. Las condiciones que se impusieron sobre $\alpha(s)$ y $\theta(s)$ para identificar cada una de las dos plantas
    se lograron al modificar la fricción de los rodamientos que unen las piezas del modelo de Simscape, permitiendo mantener un ángulo fijo cuando se está intentando ver el efecto 
    de $\widetilde{V_m}(t)$ sobre el otro.
    Las funciones de transferencia identificadas se muestran en \eqref{anguloIdent}.
\begin{equation}
    \begin{aligned}
        P_{\alpha}(s) = \frac{ \SI{5.248e-3}{}}{(0.1094s)^2 - 1}\\\label{anguloIdent} 
        P_{\theta}(s) = \frac{ \SI{-5.527e-2}{}}{s(0.3385s + 1)}
    \end{aligned}
\end{equation}
El signo negativo de la ganancia de $P_{\theta}(s)$ proviene del esquema de control utilizado.
